
O sistema eletrônico desenvolvido parte do princípio de que vários sensores podem ser utilizados para efetuar mensurações diversas, tornando a plataforma além de otimizada para a vigilância de áreas de risco, flexível inclusive para  coleta de estatísticas climáticas do local. A disposição interna dos componentes é similar a de um nanosatélite 9U, ou seja, PCBs empilhadas em um stack. A disposição interna de uma estrutura de nanosatélite 6U pode ser visualizada na figura \ref{img:nanosatelite}:

	\begin{figure}[H]
		\centering
		\includegraphics[width=0.5\textwidth]{figuras/nano}
		\caption{Estrutura para nanossatélite 6U ISIS }
		\label{img:nanosatelite}
	\end{figure}

\subsection{Áreas Contempladas} % (fold)
\label{sub:_reas_contempladas}

\subsubsubsection{Telemetria}

	A telemetria, sendo uma tecnologia que viabiliza o monitoramento, a medição ou rastreamento de uma coisa através de dados, funcionará perfeitamente para o que o projeto busca que é manter o balão sobrevoando a universidade realizando o monitoramento e ao mesmo tempo fazendo a coleta de dados sobre temperatura, umidade e também sobre alguma irregularidade na área. Essa tecnologia normalmente é feita com transmissão cabeada que possui em media 30m, ou seja, não serviria para o projeto pois possui grande risco em perda de dados pois o balão permanecerá em uma altura superior a 30m, o outro método é a transmissão sem fio (via rádio ou satélite), no caso do projeto, balão cativo, a comunicação será feita desta forma (via rádio). Existe o monitoramento em tempo real e via datalog que os dados são salvos em um cartão SD ou em pen drive, no caso do projeto o monitoramento será feito em tempo real com tempo pré determinado.

	Uma aplicação bastante utilizada da telemetria, é em balões meteorológicos desde o ano de 1920. Nesse ramo, o grande destaque é a comunicação sem fio utilizando o \textbf{SMS Relay}, que pode ser observado na figura \ref{img:telemetria}.

	\begin{figure}[H]
		\centering
		\includegraphics[width=0.3\textwidth]{figuras/telemetria}
		\caption{SMS Relay }
		\label{img:telemetria}
	\end{figure}

	O SMS Relay é um dispositivo que permite o monitoramento de uma infinita gama de equipamentos e sensores através de entradas analógicas ou digitais, e envia as informações coletadas através das entradas via SMS, ele possui também saídas de acionamento, que podem ser acionadas via SMS para ligar qualquer equipamento possibilitando uma infinidade de aplicações na automação e o mais importante é o único da categoria homologado pela ANATEL. Existem no mercado diversas soluções para monitorar algo remotamente, ou para acionar ou reiniciar equipamentos à distância, porém muitas delas não são aplicáveis na maioria das aplicações, ou seu custo inviabiliza um projeto ou torna seu uso proibitivo.

	Soluções de monitoramento baseadas internet móvel (GPRS) nem sempre apresentam um bom desempenho, pois a qualidade do serviço de internet móvel quase nunca é satisfatório onde se precisa monitorar. Já uma mensagem SMS consome muito menos dados do que o serviço de internet móvel e possui muito mais disponibilidade e estabilidade.

	Muitas soluções que utilizam radio-frequência para monitoramento remoto possuem um custo muito alto, além de implementação complicada e em muitos casos podem causar interferência em demais sistemas de comunicação o que pode causar multas e penalizações legais. Além disso há limitações de distância que a radio-frequência pode cobrir, mesmo com o uso de repetidoras e demais estruturas.

	Porém, segundo  o SMS Relay oferece uma solução confiável fabricada sobre os rígidos padrões europeus, projetada e homologada para o uso industrial e residencial, a um custo acessível quando comparado à outras soluções do mercado.

	Outro dispositivo que poderia ser utilizado na telemetria é o modulo GPRS(interface para transmissão de dados). Que pode ser observado na figura \ref{img:GPRS}.

	\begin{figure}[H]
		\centering
		\includegraphics[width=0.3\textwidth]{figuras/GPRS}
		\caption{GPRS}
		\label{img:GPRS}
	\end{figure}

	Seguem as características do GPRS ELLO Universal:

	\begin{itemize}
		\item Atualização de Firmware remoto.
		\item Programação via cabo USB.
		\item Reporta todos os eventos de qualquer painel de alarme que se comunique em Contact ID.
		\item Utiliza a tecnologia GPRS para comunicação.
		\item Saídas PGM que podem ser acionadas remotamente via GPRS.
		\item Não interfere na programação remota do painel via download.
		\item Programação realizada por software disponibilizado gratuitamente pela PPA.
		\item Supervisão anti-sabotagem e funcionamento do painel.
		\item Permite o envio de teste periódico por linha fixa.
		\item Possui detector de corte de linha telefônica - Permite o uso em locais onde não existe linha fixa.
	\end{itemize}
	% subsection _reas_contempladas (end)

\subsubsection{Sistemas de Telecomunicações}

\subsubsection{Controle e Automação}

Embora que a princípio o balão trabalhará com altitude fixa, este tem o grais de liberdade para mudar de orientação em torno dos eixos Z\textit{b}, Y\textit{b} e X\textit{b} (considera-se o sistema de referência Body Axes). O sistema de referência nos eixos do corpo tem origem geralmente no centro de massa, e utilizada para referenciar aeronaves, neste caso será aplicado à payload do balão. Estas mudanças de orientação ocasionarão rotações involuntárias de câmeras embarcadas no balão, dessa forma faz-se necessária a estabilização do movimento. O Sistema de Referência pode ser observado na figura \ref{img:sistReferencia}.

	\begin{figure}[H]
		\centering
		\includegraphics[width=0.5\textwidth]{figuras/sistReferencia}
		\caption{Sistema de Referência}
		\label{img:sistReferencia}
	\end{figure}

Tal movimento de rotação pode ser induzido pelas forças aerodinâmicas que agem no balão quando o fluxo de ar faz-se presente. O sistema de controle que seria capaz de estabilizar o sistema frente a uma perturbação seria classificado como de malha fechada, isso significa que um conjunto de sensores inerciais (acelerômetro, giroscópio) deve ser empregado para além de detectar a perturbação, verificar se o sistema de controle está sendo efetivo. Dessa forma o sistema de controle de malha fechada verifica se a saída condiz com as especificações de estabilidade do sistema, para se ter certeza de que a estabilização está sendo feita. O sistema de controle atuaria de forma intermitente enquanto a estabilização não fosse bem sucedida. Para fins de viabilidade, o sistema de controle empregado deve ser capaz de estabilizar a payload (setor de equipamentos embarcados) rapidamente, para se ter qualidade nas imagens geradas pela câmera.

Um provável atuador para o eixo ZB, ou seja, mecanismo capaz de efetuar a estabilização seria um Reaction Wheel. Um Reaction Wheel é um dispositivo frequentemente utilizado para o controle de atitude de satélites, consiste de um disco massivo acoplado a um eixo giratório. O princípio que o dispositivo usa para efetuar a estabilização é o momento de inércia do disco, dependendo da interpretação do algoritmo de controle das leituras dos sensores, sua rotação é ativada com velocidade e sentido determinados, executando-se a estabilização (anula a rotação da payload do balão no eixo). Tal atuador se encontrará no interior da payload. Como pode ser visto na figura \ref{img:reaction}.

	\begin{figure}[H]
		\centering
		\includegraphics[width=0.5\textwidth]{figuras/reaction}
		\caption{Reaction Wheels Clyde Space}
		\label{img:reaction}
	\end{figure}

	A especificação dos Reaction Wheels comerciais da compania Clyde Space se encontra na figura \ref{img:especificacao}.

	\begin{figure}[H]
		\centering
		\includegraphics[width=0.5\textwidth]{figuras/especificacao}
		\caption{Especificação de Reaction Wheels Clyde Space}
		\label{img:especificacao}
	\end{figure}

	Para a estabilização do eixo YB pode ser utilizado um trilho para mover a posição da bexiga, e dessa forma alterar o ângulo de pitch, de forma a nivelar o plano seccional horizontal da payload com o solo. Tal trilho está indicado na estrutura conceitual da payload, figura \ref{img:trilho}.

	\begin{figure}[H]
		\centering
		\includegraphics[width=0.5\textwidth]{figuras/trilho}
		\caption{Representação conceitual da payload, trilho em destaque.}
		\label{img:trilho}
	\end{figure}

	No caso o eixo X\textit{b}, a estabilização pode ser feita através da variação da altitude do balão em intervalos de distância pré-definidos. Essa variação da altitude pode ser feita através da retração e liberação do cabo na carretilha em solo. A instabilidade no eixo Z\textit{b} não afetará significativamente a qualidade da imagem, desde que a estabilização nos outros dois eixos seja efetiva.

	Uma provável automação efetuada pelo balão será a avaliação de sua própria segurança. Por meio de sensores de tensão no cabo (dinamômetro) preso na estrutura da figura 5, se esta aumentar acima de um nível critico, este será automaticamente recolhido por meio do rotor motorizado em solo, e a estação de solo será informada. Assim que o sensor em solo sinalizar normalidade na velocidade do vento este será novamente elevado.

	\begin{figure}[H]
		\centering
		\includegraphics[width=0.5\textwidth]{figuras/parteInferior}
		\caption{Representação conceitual da parte inferior da payload, destinada à anexação do fio preso ao solo.}
		\label{img:parteInferior}
	\end{figure}

	Mais uma automação essencial será a sua elevação e retração automática para o período de monitoração determinada.

\subsubsection{Câmeras}
	A câmera ideal precisaria ter aproximadamente 42 LEDS para o uso em infravermelho, uma vez que funcionará a noite e cada LED ilumina aproximadamente 1m, e tendo em vista que o valor aproximado que cada câmera irá filmar é de 40 m na horizontal. As maiorias das câmeras de segurança utilizam lentes na faixa de 3 mm e com esse valor é possível ter uma boa visão panorâmica, mas como o objetivo do uso da câmera é obter imagens nítidas e com  muito detalhes de faces de pessoas ou placas de carro, seria necessário o uso de lentes maiores como 6 mm, a qual terá mais nitidez e menos visão panorâmica.

	Outra característica que aperfeiçoaria o uso da câmera seria ter um conversor AD acoplado, pois não seria necessário o uso de um conversor para o armazenamento e por ser preciso liga-lo ao raspberry. As câmeras que possuem esse tipo de conversor câmeras DSLR (Digita Single-Lens Reflex), as quais são câmeras profissionais que oferecem lentes substituíveis, grandes sensores de imagens e etc..

	Além disso, nossa câmera ideal precisa ter uma compressão de vídeo H264, o qual de forma resumida tem a capacidade de compactar de forma avançada com uma experiência de vídeo superior a uma taxa de bits baixa e se adequa a qualquer plataforma, portátil a alta definição.

	A câmera VM 3130 IR é uma opção para este projeto, pois é possível encontra-la com lentes de 6 mm, com um alcance de IR de 30 m, além de ter uma resolução consideravelmente boa de 960 linhas na horizontal e 720 de resolução real.

\subsubsection{Sensores}
	No estudo da automação em sistemas industriais, é preciso determinar as condições do sistema. É necessário obter os valores das variáveis físicas do ambiente a ser monitorado, e este é o trabalho dos sensores.
	Os sensores são dispositivos sensíveis a alguma forma de energia do ambiente que pode ser luminosa, térmica, cinética, relacionando informações sobre uma grandeza que precisa ser medida, como por exemplo a temperatura, aceleração, posição e etc.

\paragraph{Temperatura}
	Para a elaboração desse projeto vai ser utilizado o sensor eletrônico LM 35 como sensor de temperatura, que atua de forma prática e objetiva. O LM 35 é um sensor de precisão, fabricado pela National Semiconductor, a tensão de saída será linear e relativa à temperatura em que se encontra no momento em que for alimentado por uma tensão de 4-20V e GND e drena apenas \SI{60}{\micro\metre\ampere}. O valor de saída é de \SI{10}{\milli\volt} para cada grau Celsius, o que dá uma boa vantagem por não ser necessário uma calibração como nos sensores de temperatura em Kelvin. O encapsulamento mais comum do LM 35 é o TO-92.~\cite{LM35}

	Esses sensores podem ser utilizados em conjunto com o circuito integrado LM 3914, que funciona como comparador e apresenta um drive-display. Este CI monitora a escala de tensão analógica e controla a saída em função de LEDs que integram um display analógico linear. Esse CI pode ser usado para uma questão de segurança, por exemplo ao atingir uma temperatura limite do produto, dispara suas saídas.~\cite{LM3914}

	Estes sensores podem ser usados também para monitorar a temperatura dos outros equipamentos eletrônicos que compõem o projeto. São sensores micro eletrônicos, logo não vão influenciar no peso do balão.

\paragraph{Inercia}
	Sensores de inércia são os sensores para captar algum movimento na estrutura. Para manter um monitoramento destes requisitos vamos usar o sensor GY-80 10 DOF, pois é um dispositivo que é capaz de medir uma boa parte dos parâmetros necessários, é um equipamento compacto e de baixo custo, possui um peso de 5g e dimensões de 25.8 x 16.8mm. É uma unidade de medida inercial, capaz de informar e medir a velocidade, orientação de forças gravitacionais a partir de um acelerômetro, um giroscópio, um magnetômetro e um barômetro. São 3 eixos do giroscópio L3G4200D, 3 eixos do acelerômetro ADXL345, 3 eixos do magnetômetro HMC5883L e mais o sensor de pressão BMP085.

	O magnetômetro HMC5883L mede o campo magnético e funciona como uma bússola digital, podendo ser usado com um Arduíno para indicar o norte geográfico da Terra. Pode operar em temperaturas de -30ºC à 85ºC, é alimentado com uma tensão entre 2 a 3.6V e um corrente de 100uA.~\cite{HMC5883L}

	O Giroscópio L3G4200D tem capacidade para 3 eixos, bem como 3 níveis de sensibilidade. Os dados das velocidades angulares podem ser obtidos através da comunicação I2C. Opera em temperaturas de -40ºc à 85ºC, alimentação de 2.4V a 3.6V e corrente de 6.1uA, faixa do giroscópio de: 250/500/2000 º/s.~\cite{L3G4200D}

	O Acelerômetro de 3 eixos ADXL345 possui alta resolução com baixo consumo de energia. Os dados também são obtidos por comunicação I2C. Opera em temperaturas de -40ºC à 85ºC, tensão de alimentação entre 2.0 e 3.6V, suporta impactos de até 10000g.~\cite{ADXL345}

	O barômetro BMP085 consiste em um sensor piezo-resistivo, um sistema de conversão A/D e uma unidade de controle com EEPROM e interface I2C. Realiza a medição da pressão atmosférica. Ele pode operar em temperaturas entre -40ºC e 85ºC, é alimentado por uma tensão de 1.8 à 3.6V e corrente de 3 à 12uA, pressão medida em hPa.~\cite{BMP085}

\paragraph{Umidade}
	A umidade é outro fator que pode influenciar no comportamento do balão, já que o balão será movido a gás. Para o sensoreamento dessa umidade vai ser utilizado o sensor DHT11, que é um sensor que permite fazer a leitura da umidade entre 20 a 90\% e pode ser utilizado juntamente com Arduíno. É alimentado com tensão de 3-5V e corrente de 200uA, possui tempo de resposta de 2 segundos, precisão de medição de umidade de mais ou menos 5.0\% UR e tem dimensões de 23 x 12 x 5 mm.[7]

\subsubsection{Conversor Analógico/Digital}
      \paragraph{Definição}

      Os sinais que existem no mundo real são analógicos e, por essa razão, esses sinais devem ser transformados em digitais por meio de um conversor, para que possam ser manipulados pelo equipamento digital.

      Um conversor analógico/digital, também conhecido com A/D ou ADC, é um dispositivo que gera um sinal digital a partir de um analógico. Normalmente, esse sinal analógico é o valor de uma tensão ou corrente. Esses instrumentos são utilizados na interface entre dispositivos digitais e analógicos. Uma das grandes vantagens de se obter um valor digital é a capacidade de compactação de dados, podendo diminuir o tamanho do arquivo, economizando espaço na largura de banda, que é uma dos limitadores do conversor. Geralmente, é utilizando um filtro passa-baixas antes do conversor A/D, com o objetivo de evitar que amplitudes de alta frequência apareçam na entrada do conversor.

      \paragraph{Funcionamento}

      Os microcontroladores processam dados obtidos por sensores. No entanto, na saída dos sensores é encontrado valores analógicos, logo é necessário transformá-los em valores digitais. Então, para executar essa atividade, é preciso do conversor A/D, que interfaceiam os dispositivos de medidas, e o microcontrolador, como mostrado na figura \ref{img:conversorAD}:

      \begin{figure}[h]
        \raggedleft
        \includegraphics[width=0.87\textwidth]{figuras/conversorAD}
        \caption{Conversão A/D}
        \label{img:conversorAD}
      \end{figure}

			Nesses conversores, quanto maior o número bits de saída, melhor ele será. Por exemplo, um conversor que tem uma saída de quatro bits possui dezesseis degraus de indicação, ou seja, pode definir uma escala de dezesseis valores diferentes, conforme mostrado na figura \ref{img:escalaLEDs}.

			\begin{figure}[h]
        \centering
        \includegraphics[width=0.75\textwidth]{figuras/escalaLEDs}
        \caption{Escala de LEDs}
        \label{img:escalaLEDs}
      \end{figure}

			Se o circuito converte sinais na faixa de 0V a 1V, é preciso ter cuidado para que os sensores usados trabalhem nessa faixa. Um amplificador operacional pode ter um ganho programado para evitar esses problemas. Então, as saídas terão um número n de pinos nas quais as saídas nos níveis lógicos 0 ou 1 são obtidos conforme a tensão de entrada.

			\begin{figure}[h]
        \centering
        \includegraphics[width=0.5\textwidth]{figuras/conexaoPC}
        \caption{Conexão direta com o computador}
        \label{img:conxaoPC}
      \end{figure}

      \paragraph{Nanoshield ADC}

			O Nanoshield [fig. \ref{img:nanoshield}] é um conversor A/D de grande resolução. Implementado com o CI ADS1115, ele é ideal para aplicações de leitura de sensores industriais de temperatura, umidade, monitoramento de baterias, tensões de alimentação ou qualquer projeto que precisa de um conversor A/D de alta qualidade.

			\begin{figure}[h]
        \centering
        \includegraphics[width=0.5\textwidth]{figuras/nanoshield}
        \caption{Nanoshield}
        \label{img:nanoshield}
      \end{figure}

			Esse conversor possui quatro entradas analógicas independentes, podendo ser utilizadas para medir tensões absolutas. Essas entradas pode ser facilmente configuradas para ler tensões de 0 a 10V. E isso é feito por meio de um divisor resistivo que já é implementado de fábrica, precisando apenas ser habilitado fechando-se jumpers de solda na placa.

			\begin{table}[h!]
				\centering
				\begin{tabular}{ccc}
					\toprule
						\textbf{Jumper} & \textbf{Aberto} & \textbf{Fechado}\\
					\midrule
						JP0 & A0: 0 a 5V & A0: a 10V\\
						JP1 & A1: 0 a 5V & A1: a 10V\\
						JP2 & A2: 0 a 5V & A2: a 10V\\
						JP3 & A3: 0 a 5V & A3: a 10V\\
					\bottomrule
				\end{tabular}
				\caption{Jumpers}
				\label{tab:jumpers}
			\end{table}

			\begin{figure}[h!]
        \raggedleft
        \includegraphics[width=0.87\textwidth]{figuras/jumpers}
        \caption{Jumpers de habilitação para leitura de sensores de 0 a 10V}
        \label{img:jumpers}
      \end{figure}

			O conversor pode se comunicar com um processador através de um barramento $I^2$C que utiliza dois pinos. Na figura 6, é apresentado um diagrama de bloco  mostrando, de maneira mais detalhada, o funcionamento do módulo.

			\begin{figure}[h!]
				\centering
				\includegraphics[width=0.5\textwidth]{figuras/diagramaNanoshield}
				\caption{Diagrama de blocos do Nanoshield ADC}
				\label{img:diagramaNanoshield}
			\end{figure}

			Em nosso projeto, é preciso ter precisões nas medidas necessárias e, por isso, um conversor de 16 bits é importantíssimo. Além disso, quatro entradas por módulo, 860 amostras, amplificador interno de baixas amplitudes, filtros RC em todos os canais e entradas analógicas protegidas contra ligações acidentais até $\pm$24V.

			No caso do Balão Monitorado, as medições de temperatura passarã por este conversor. Como se trata de um sensor que trabalha na faixa de operação do conversor Nanoshield, ele será a melhor escolha para cumprir os requisitos do projeto.
