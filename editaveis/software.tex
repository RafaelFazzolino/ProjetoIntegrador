Tendo em vista que o balão necessitará armazenar informações das filmagens, apresentar as informações, tanto em tempo real, quanto a demanda, será necessário um desenvolvimento de um software próprio para que suporte algumas características do projeto.

\subsection{Visão geral do produto}

O software em questão será subdividido em duas partes, o software que estará no balão, embarcado, e o software que estará em terra, central de controle.

\subsubsection{Software embarcado}

O software embarcado irá se basear em apenas recolher os dados dos sensores e câmeras e enviar para a central de controle.

O envio dos dados será realizado utilizando um serviço \textit{restfull} para simplificar e reduzir os custos e o peso de hardware no balão, deixando a responsabilidade de processar, catalogar e armazenar os dados enviados com o lado servidor que se encontrará em terra.

\subsubsection{Central de controle}

O software que estará em terra será o responsável por recolher os dados enviados pelo software embarcado e armazenar essas informações no banco de dados, alem de ser responsável pela redundância de dados e resiliência do sistema como um todo, evitando assim falhas previstas e tornando possível um maior tempo de operação direta.

\subsubsection{Resumo das capacidades}

Nesta seção serão descritos o que será delegado a cada parte do software e o que está dentro ou fora das capacidades do mesmo.

\paragraph{Software embarcado}

O software embarcado será responsável pelo recolhimento dos dados de cada sensor, compilando-os em uma pacote para que este possa ser enviado ao servidor, devendo ser capaz de identificar se o serviço ao qual irá consumir está disponível, e caso não esteja possa salvar as informações durante um curto período de tempo e enviar ao servidor quando o serviço estiver novamente funcional.

Além da capacidade de perceber se o serviço está funcional ou não, o software embarcado deve ser capaz de verificar um mal funcionamento em algum sensor do balão para que as devidas providências possam ser tomadas.

\paragraph{Central de controle}

O software em terra será o coração do controle informatizado, sendo responsável por servir o software embarcado com um serviço \textit{restfull} para possibilitar a comunicação rápida e direta entre as duas partes.

Além do serviço, a central de controle será responsável pela redundância de dados para segurança da informação e detecção de falhas para ter a possibilidade de correção automática.

\subsection{Requisitos}

Os requisitos do software são informações das quais serão levantados caso de uso para o sistema, que por sua vez nortearão o desenvolvimento das funcionalidades delimitando o que cada uma deve ou não conter.

\begin{itemize}
  \item O sistema deve ser capaz de interfaciar com diversos sensores, sendo capaz de incluir novos caso haja a necessidade.
  \item O sistema deve realizar um \textit{backup} das informações dos vídeos a cada 2 horas.
  \item O sistema deve ser capaz de identificar mal funcionamento nos sensores aos quais estabelece interface.
  \item O sistema deve avisar quando forem detectados falhas ou mal funcionamento tanto na unidade terrestre quanto na unidade aérea.
\end{itemize}
