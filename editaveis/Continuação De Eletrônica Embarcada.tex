\subsubsection{Câmeras}
\indent
\par
A câmera ideal precisaria ter aproximadamente 42 LEDS para o uso em infravermelho, uma vez que funcionará a noite e cada LED ilumina aproximadamente 1m, e tendo em vista que o valor aproximado que cada câmera irá filmar é de 40 m na horizontal. As maiorias das câmeras de segurança utilizam lentes na faixa de 3 mm e com esse valor é possível ter uma boa visão panorâmica, mas como o objetivo do uso da câmera é obter imagens nítidas e com  muito detalhes de faces de pessoas ou placas de carro, seria necessário o uso de lentes maiores como 6 mm, a qual terá mais nitidez e menos visão panorâmica.

\indent
\par
Outra característica que aperfeiçoaria o uso da câmera seria ter um conversor AD acoplado, pois não seria necessário o uso de um conversor para o armazenamento e por ser preciso liga-lo ao raspberry. As câmeras que possuem esse tipo de conversor câmeras DSLR (Digita Single-Lens Reflex), as quais são câmeras profissionais que oferecem lentes substituíveis, grandes sensores de imagens e etc..
\indent
\\\par

Além disso, nossa câmera ideal precisa ter uma compressão de vídeo H264, o qual de forma resumida tem a capacidade de compactar de forma avançada com uma experiência de vídeo superior a uma taxa de bits baixa e se adequa a qualquer plataforma, portátil a alta definição.
\indent
\\\par
A câmera VM 3130 IR é uma opção para este projeto, pois é possível encontra-la com lentes de 6 mm, com um alcance de IR de 30 m, além de ter uma resolução consideravelmente boa de 960 linhas na horizontal e 720 de resolução real.

\subsubsection{Sensores}
\indent
\par
No estudo da automação em sistemas industriais, é preciso determinar as condições do sistema. É necessário obter os valores das variáveis físicas do ambiente a ser monitorado, e este é o trabalho dos sensores.
\indent
\\\par	
Os sensores são dispositivos sensíveis a alguma forma de energia do ambiente que pode ser luminosa, térmica, cinética, relacionando informações sobre uma grandeza que precisa ser medida, como por exemplo a temperatura, aceleração, posição e etc.

\subsubsection{Sensores - Temperatura}
\indent
\par
Para a elaboração desse projeto vai ser utilizado o sensor eletrônico LM 35 como sensor de temperatura, que atua de forma prática e objetiva. O LM 35 é um sensor de precisão, fabricado pela National Semiconductor, a tensão de saída será linear e relativa à temperatura em que se encontra no momento em que for alimentado por uma tensão de 4-20V e GND e drena apenas 60μ A. O valor de saída é de 10mV para cada grau Celsius, o que dá uma boa vantagem por não ser necessário uma calibração como nos sensores de temperatura em Kelvin. O encapsulamento mais comum do LM 35 é o TO-92.[1]
\indent
\\\par
Esses sensores podem ser utilizados em conjunto com o circuito integrado LM 3914, que funciona como comparador e apresenta um drive-display. Este CI monitora a escala de tensão analógica e controla a saída em função de LEDs que integram um display analógico linear. Esse CI pode ser usado para uma questão de segurança, por exemplo ao atingir uma temperatura limite do produto, dispara suas saídas.[2]
\indent
\\\par
Estes sensores podem ser usados também para monitorar a temperatura dos outros equipamentos eletrônicos que compõem o projeto. São sensores micro eletrônicos, logo não vão influenciar no peso do balão.

\subsubsection{Sensores - Inercia}
\indent
\par
Sensores de inércia são os sensores para captar algum movimento na estrutura. Para manter um monitoramento destes requisitos vamos usar o sensor GY-80 10 DOF, pois é um dispositivo que é capaz de medir uma boa parte dos parâmetros necessários, é um equipamento compacto e de baixo custo, possui um peso de 5g e dimensões de 25.8 x 16.8mm. É uma unidade de medida inercial, capaz de informar e medir a velocidade, orientação de forças gravitacionais a partir de um acelerômetro, um giroscópio, um magnetômetro e um barômetro. São 3 eixos do giroscópio L3G4200D, 3 eixos do acelerômetro ADXL345, 3 eixos do magnetômetro HMC5883L e mais o sensor de pressão BMP085.

\indent
\\\par
O magnetômetro HMC5883L mede o campo magnético e funciona como uma bússola digital, podendo ser usado com um Arduíno para indicar o norte geográfico da Terra. Pode operar em temperaturas de -30ºC à 85ºC, é alimentado com uma tensão entre 2 a 3.6V e um corrente de 100uA.[3]

\indent
\par
O Giroscópio L3G4200D tem capacidade para 3 eixos, bem como 3 níveis de sensibilidade. Os dados das velocidades angulares podem ser obtidos através da comunicação I2C. Opera em temperaturas de -40ºc à 85ºC, alimentação de 2.4V a 3.6V e corrente de 6.1uA, faixa do giroscópio de: 250/500/2000 º/s.[4]

\indent
\par
O Acelerômetro de 3 eixos ADXL345 possui alta resolução com baixo consumo de energia. Os dados também são obtidos por comunicação I2C. Opera em temperaturas de -40ºC à 85ºC, tensão de alimentação entre 2.0 e 3.6V, suporta impactos de até 10000g.[5]

\indent
\par
O barômetro BMP085 consiste em um sensor piezo-resistivo, um sistema de conversão A/D e uma unidade de controle com EEPROM e interface I2C. Realiza a medição da pressão atmosférica. Ele pode operar em temperaturas entre -40ºC e 85ºC, é alimentado por uma tensão de 1.8 à 3.6V e corrente de 3 à 12uA, pressão medida em hPa.[6]

\subsubsection{Sensores - Umidade}
\indent
\par
A umidade é outro fator que pode influenciar no comportamento do balão, já que o balão será movido a gás. Para o sensoreamento dessa umidade vai ser utilizado o sensor DHT11, que é um sensor que permite fazer a leitura da umidade entre 20 a 90\% e pode ser utilizado juntamente com Arduíno. É alimentado com tensão de 3-5V e corrente de 200uA, possui tempo de resposta de 2 segundos, precisão de medição de umidade de mais ou menos 5.0\% UR e tem dimensões de 23 x 12 x 5 mm.[7]
\\
\section{Referências Bibliográficas}
\indent \par
[1] Texas Instruments. Datasheet: LM35 Precision Centigrade Temperature Sensors. Eletronic Publication, 1999.
\indent \par [2] Texas Instruments. Datasheet: LM3914 Dot/Bar Display Driver. Eletronic Publication, 2000.
\indent \par [3] Honeywell. Datasheet: HMC5883L 3-Axis Digital Compass IC. Eletronic Publication, 2010.
\indent \par [4] ST Microeletronics. Datasheet: L3G4200D MEMS Motion Sensor. Eletronic Publication, 2010.
\indent \par [5] Analog Devices. Datasheet: ADXL345 Digital Accelerometer. Eletronic Publication, 2009.
\indent \par [6] Bosch Sensortec. Datasheet: BMP085 Digital pressure sensor. Eletronic Publication, 2008.
\indent \par [7] Aosong Eletronics. Datasheet: DHT11 Temperature and Humidity module.
