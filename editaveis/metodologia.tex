A metodologia de gerenciamento de projetos utilizada é o SCRUM. O SCRUM é uma metodologia ágil para gestão de projetos na área de software \cite{devagil}. O SCRUM, no contexto da disciplina de Projeto Integrador 1, foi adaptado para o gerenciamento de um projeto de engenharia.

“Parece que alguns percebem o SCRUM como um framework aplicável apenas para o desenvolvimento de software. No entanto, ao se olhar mais de perto o framework SCRUM, nota-se que os seus pilares não possuem qualquer ligação direta ao desenvolvimento de software. Portanto, é possível concluir que o SCRUM pode ser usado em qualquer tipo de desenvolvimento de produtos“ \cite{scrumineng} O QUE É ISSO, EDU?

No SCRUM, as funcionalidades do produto são organizadas em dois artefatos, o backlog de produto e o backlog da sprint. O backlog do produto possui uma visão mais abrangente da funcionalidade, enquanto que no backlog da sprint, essas funcionalidades são refinadas e melhor descritas.

Uma das principais características do SCRUM é a entrega contínua, realizada em um períodos de duas a quatro semanas. Este período em que se desenvolve e entrega um conjunto de funcionalidades é denominado sprint.

No inicio de cada sprint, um conjunto de funcionalidades são selecionadas do backlog de produto, refinadas e alocadas no backlog da sprint. O objetivo da sprint é portanto, consumir o esse backlog.

Para conseguir refinar a as funcionalidades de forma adequada e planejar com maior seguraça a sprint, é realizada uma reunião de planejamento da sprint, em que todos os membros da equipe devem estar presentes.

Além da reunião de planejamento da sprint, é realizado uma reunião diária curta para que toda a equipe tenha ciência do que está acontecendo no projeto, evitando ruídos ou o distanciamento dos integrantes.

Ao final das sprints é feito duas reuniões, a reunião de revisão de sprint, em que apresenta-se o objetivos alcançados na sprint, incluindo demostrações do que foi desenvolvido; e a retrospectiva da sprint, em que os membros da equipe discutem os problemas enfrentados durante a sprint, como melhorar a próxima sprint, e o que de positivo aconteceu.
